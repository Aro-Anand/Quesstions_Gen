\documentclass[12pt]{article}
\usepackage{amsmath}
\usepackage{amssymb}
\usepackage{geometry}
\geometry{margin=1in}
\title{Question Paper}
\author{Class 10 - Math}
\date{\today}
\begin{document}
\maketitle

\section*{Chapter: Algebra}
\subsection*{Topic: Quadratic Equations}
\textbf{Difficulty Level:} 5/5 \\
\textbf{Question Type:} Objective

\begin{enumerate}

\item Find the nature of the roots of \(x^2 - 6x + 9 = 0\).

\begin{enumerate}
  \item Two distinct real roots \(x = 0, x = 6\)
  \item One real repeated root \(x = 3\)
  \item No real roots
  \item Complex roots
\end{enumerate}

\item For which real values of \(k\) does \(x^2 + kx + 5 = 0\) have a repeated real root?

\begin{enumerate}
  \item k = -2\sqrt{5}
  \item k = 0
  \item k = \pm 2\sqrt{5}
  \item no real value of k
\end{enumerate}

\item Solve by factoring: \(3x^2 - 12x + 9 = 0\).

\begin{enumerate}
  \item \(x = 1\text{ and }x = 3\)
  \item \(x = 0\text{ and }x = 3\)
  \item \(x = -1\text{ and }x = -3\)
  \item \(x = 3\text{ only}\)
\end{enumerate}

\item Using the quadratic formula, solve \(2x^2 + 3x - 2 = 0\).

\begin{enumerate}
  \item \(x = -2,\; x = \tfrac{1}{2}\)
  \item \(x = 2,\; x = -\tfrac{1}{2}\)
  \item \(x = 1,\; x = -2\)
  \item \(x = -1,\; x = 2\)
\end{enumerate}

\item If \(a = 2, b = -5, c = 3\), determine the roots of \(ax^2 + bx + c = 0\).

\begin{enumerate}
  \item \(x = 1\text{ and }x = \tfrac{3}{2}\)
  \item \(x = -1\text{ and }x = \tfrac{3}{2}\)
  \item \(x = \tfrac{1}{2}\text{ and }x = 3\)
  \item \(x = -\tfrac{1}{2}\text{ and }x = -\tfrac{3}{2}\)
\end{enumerate}

\item The quadratic \(x^2 - 4x + k = 0\) has real roots that are both positive. Find the range of \(k\).

\begin{enumerate}
  \item \(k \le 0\)
  \item \(0 < k \le 4\)
  \item \(k < 4\)
  \item \(k \ge 4\)
\end{enumerate}

\item Complete the square for \(x^2 + 4x - 5 = 0\).

\begin{enumerate}
  \item \((x+2)^2 - 9 = 0\)
  \item \((x-2)^2 - 9 = 0\)
  \item \((x+2)^2 + 9 = 0\)
  \item \((x-2)^2 + 9 = 0\)
\end{enumerate}

\item If the roots are equal, the discriminant satisfies \(D = 0\).

\begin{enumerate}
  \item \(D > 0\)
  \item \(D = 0\)
  \item \(D < 0\)
  \item \(D\) is a perfect square only if \(a = 1\)
\end{enumerate}

\item Solve by factoring: \(9x^2 - 24x + 16 = 0\).

\begin{enumerate}
  \item \(x = \tfrac{4}{3}\)
  \item \(x = 0\)
  \item \(x = \tfrac{2}{3}\)
  \item \(x = -\tfrac{4}{3}\)
\end{enumerate}

\item The equation \(2x^2 + 4x + 5 = 0\) has which type of roots?

\begin{enumerate}
  \item Two real distinct roots
  \item One real repeated root
  \item No real real roots (complex roots)
  \item Infinitely many roots
\end{enumerate}

\item If the roots of \(x^2 - 7x + 12 = 0\) are \(r_1\) and \(r_2\), what is \(r_1 r_2\)?

\begin{enumerate}
  \item \(7\)
  \item \(12\)
  \item \(-12\)
  \item \(49\)
\end{enumerate}

\item If the roots are \(\tfrac{1}{2}\) and 4, the quadratic with leading coefficient 2 is which of the following?

\begin{enumerate}
  \item \(2x^2 - 9x + 4 = 0\)
  \item \(2x^2 - 9x + 8 = 0\)
  \item \(2x^2 - 10x + 4 = 0\)
  \item \(2x^2 - 9x + 2 = 0\)
\end{enumerate}

\item Which of the following is the equation with leading coefficient \(2\) having roots \(3\) and \(-4\)?

\begin{enumerate}
  \item \(2x^2 + 2x - 24 = 0\)
  \item \(2x^2 + 8x - 24 = 0\)
  \item \(2x^2 - 2x - 24 = 0\)
  \item \(2x^2 + 2x + 24 = 0\)
\end{enumerate}

\item Rewrite \(x^2 - 4x - 5 = 0\) in completed square form.

\begin{enumerate}
  \item \((x-2)^2 - 9 = 0\)
  \item \((x+2)^2 - 9 = 0\)
  \item \((x-2)^2 + 9 = 0\)
  \item \((x+2)^2 + 9 = 0\)
\end{enumerate}

\item If the discriminant \(D = b^2 - 4ac = 0\), the roots are:

\begin{enumerate}
  \item two distinct real roots
  \item a single real repeated root
  \item complex roots
  \item not defined
\end{enumerate}

\item Solve by factoring: \(6x^2 - 13x + 6 = 0\).

\begin{enumerate}
  \item \(x = \tfrac{2}{3}, \; x = \tfrac{3}{2}\)
  \item \(x = \tfrac{1}{3}, \; x = 6\)
  \item \(x = -\tfrac{2}{3}, \; x = -\tfrac{3}{2}\)
  \item \(x = 1, \; x = 6\)
\end{enumerate}

\item If \(r_1 + r_2 = 5\) and \(r_1 r_2 = 6\), which equation has these roots?

\begin{enumerate}
  \item \(x^2 - 5x + 6 = 0\)
  \item \(x^2 + 5x + 6 = 0\)
  \item \(x^2 - 5x - 6 = 0\)
  \item \(x^2 + 5x - 6 = 0\)
\end{enumerate}

\item Compute the roots of \(4x^2 - 4x - 1 = 0\) using the quadratic formula.

\begin{enumerate}
  \item \(x = (1 \pm \sqrt{2})/2\)
  \item \(x = (1 \pm \sqrt{3})/2\)
  \item \(x = (-1 \pm \sqrt{2})/2\)
  \item \(x = (1 \pm \sqrt{5})/2\)
\end{enumerate}

\item Solve by factoring: \(x^2 - 2x - 3 = 0\).

\begin{enumerate}
  \item \(x = 3\text{ and }x = -1\)
  \item \(x = 1\text{ and }x = 3\)
  \item \(x = -1\text{ and }x = -3\)
  \item \(x = 2\text{ and }x = -1\)
\end{enumerate}

\item Solve by completing the square: \(x^2 + \tfrac{3}{2}x - 2 = 0\).

\begin{enumerate}
  \item \(x = \dfrac{-3 \pm \sqrt{41}}{4}\)
  \item \(x = \dfrac{-3 \pm \sqrt{41}}{2}\)
  \item \(x = \dfrac{-3 \pm \sqrt{41}}{4}\)
  \item \(x = \dfrac{3 \pm \sqrt{41}}{4}\)
\end{enumerate}

\item For which real value of \(k\) does \(x^2 + kx + 1 = 0\) have equal roots?

\begin{enumerate}
  \item \(k = 0\)
  \item \(k = \pm 2\)
  \item \(k = \pm 2i\)
  \item No real value of k
\end{enumerate}

\item The equation \(8x^2 - 2x - 3 = 0\) has roots:

\begin{enumerate}
  \item \(x = \tfrac{3}{4}\text{ and }x = -\tfrac{1}{2}\)
  \item \(x = \tfrac{3}{4}\text{ and }x = \tfrac{1}{2}\)
  \item \(x = -\tfrac{3}{4}\text{ and }x = \tfrac{1}{2}\)
  \item \(x = -\tfrac{3}{4}\text{ and }x = -\tfrac{1}{2}\)
\end{enumerate}

\item Which is the axis of symmetry for the parabola \(y = x^2 - 6x + 5\)?

\begin{enumerate}
  \item \(x = 2\)
  \item \(x = 3\)
  \item \(x = -3\)
  \item \(x = -2\)
\end{enumerate}

\item If \(r_1, r_2\) are the roots of \(2x^2 - 8x + 6 = 0\), which statement is true?

\begin{enumerate}
  \item \(r_1 + r_2 = 4,\; r_1 r_2 = 3\)
  \item \(r_1 + r_2 = 4,\; r_1 r_2 = -3\)
  \item \(r_1 + r_2 = -4,\; r_1 r_2 = 3\)
  \item \(r_1 + r_2 = 2,\; r_1 r_2 = 3\)
\end{enumerate}

\item Compute the roots of \(x^2 + 6x + 9 = 0\).

\begin{enumerate}
  \item \(x = -3\text{ (repeated root)}\)
  \item \(x = 3\text{ and }x = -3\)
  \item \(x = -9\)
  \item No real roots
\end{enumerate}

\item Which of the following is the quadratic formula to find the roots of \(ax^2 + bx + c = 0\)?

\begin{enumerate}
  \item \(x = \dfrac{-b \pm \sqrt{b^2 - 4ac}}{2a}\)
  \item \(x = \dfrac{b \pm \sqrt{b^2 - 4ac}}{2a}\)
  \item \(x = \dfrac{-b \pm \sqrt{b^2 - ac}}{a}\)
  \item \(x = \dfrac{-b \pm \sqrt{a^2 - 4bc}}{2c}\)
\end{enumerate}

\item If the roots of \(x^2 + 4x + 5 = 0\) are complex, what is the sum of the roots?

\begin{enumerate}
  \item \(-4\)
  \item \(4\)
  \item \(0\)
  \item \(2\)
\end{enumerate}

\item Solve \(x^2 - 9 = 0\).

\begin{enumerate}
  \item \(x = 9\text{ or }x = -9\)
  \item \(x = 0\text{ or }x = 9\)
  \item \(x = 3\text{ or }x = -3\)
  \item \(x = 1\text{ or }x = -1\)
\end{enumerate}

\item Find the roots of \(x^2 + 2x + 2 = 0\).

\begin{enumerate}
  \item \(-1 \pm i\)
  \item \(-1 \pm \sqrt{3}i\)
  \item \(1 \pm i\)
  \item \(-1 \pm 2i\)
\end{enumerate}

\item A quadratic with integer coefficients has roots \(3\) and \(-4\). Which is the equation with leading coefficient 1?

\begin{enumerate}
  \item \(x^2 - (3 + (-4))x + (3)(-4) = 0\)
  \item \(x^2 - x - 12 = 0\)
  \item \(x^2 - (-1)x - 12 = 0\)
  \item \(x^2 + x - 12 = 0\)
\end{enumerate}

\item Which is the factorization of \(x^2 - 5x + 6\)?

\begin{enumerate}
  \item \((x-2)(x-3)\)
  \item \((x-1)(x-6)\)
  \item \((x+2)(x-3)\)
  \item \((x-5)(x-1)\)
\end{enumerate}

\item Compute the discriminant of \(3x^2 + 4x + 1 = 0\) and identify the nature of the roots.

\begin{enumerate}
  \item \(D = 4\,;\; 2\text{ real roots}\)
  \item \(D = -8\,;\; \text{complex roots}\)
  \item \(D = 4\,;\; \text{repeated root}\)
  \item \(D = 0\,;\; \text{repeated root}\)
\end{enumerate}

\item Which equation has two equal real roots?

\begin{enumerate}
  \item \(x^2 - 2x + 2 = 0\)
  \item \(x^2 - 4x + 4 = 0\)
  \item \(x^2 + 4x + 4 = 0\)
  \item \(\text{Both B and C}\)
\end{enumerate}

\item Find the vertex of the parabola \(y = 2x^2 - 8x + 3\).

\begin{enumerate}
  \item \((2,-5)\)
  \item \((2,-1)\)
  \item \((4,-3)\)
  \item \((0,3)\)
\end{enumerate}

\item If the roots of \(x^2 + px + q = 0\) are real and equal, which condition must hold for \(p\) and \(q\)?

\begin{enumerate}
  \item \(p^2 = 4q\)
  \item \(p^2 = -4q\)
  \item \(p = 2q\)
  \item \(p = -2q\)
\end{enumerate}

\item Solve by completing the square: \(x^2 + 6x + 13 = 0\).

\begin{enumerate}
  \item \(x = -3 \pm 2i\)
  \item \(x = -3 \pm i\)
  \item \(x = -3 \pm 3i\)
  \item \(x = -3 \pm 4i\)
\end{enumerate}

\item Which method is most suitable for solving \(x^2 + 4x + 5 = 0\) without a calculator?

\begin{enumerate}
  \item Factoring
  \item Completing the square
  \item Graphical method
  \item Newton's method
\end{enumerate}

\item The discriminant of \(3x^2 + 2x + 3\) is:

\begin{enumerate}
  \item \(-27\)
  \item \(4\)
  \item \(0\)
  \item \(27\)
\end{enumerate}

\item If \(2x^2 + kx - 3 = 0\) has integer roots, which of the following could be a possible value of \(k\)?

\begin{enumerate}
  \item \(k = 5\)
  \item \(k = -5\)
  \item \(k = 1\)
  \item \(k = -1\)
\end{enumerate}

\item The roots of \(x^2 - 2x + 2 = 0\) are:

\begin{enumerate}
  \item \(-1 \pm i\)
  \item \(1 \pm i\)
  \item \(-1 \pm 2i\)
  \item \(1 \pm 2i\)
\end{enumerate}

\item Which is the vertex form of \(y = x^2 - 6x + 5\)?

\begin{enumerate}
  \item \(y = (x-3)^2 + -4\)
  \item \(y = (x-3)^2 - 4\)
  \item \(y = (x+3)^2 - 4\)
  \item \(y = (x-3)^2 + 4\)
\end{enumerate}

\item If \(r_1, r_2\) are the roots of \(x^2 - 5x + 5 = 0\), find \(r_1^3 + r_2^3\).

\begin{enumerate}
  \item \(50\)
  \item \(35\)
  \item \(25\)
  \item \(45\)
\end{enumerate}

\item Which of the following is a quadratic equation?

\begin{enumerate}
  \item \(x^3 - x = 0\)
  \item \(3x^2 + 2x - 5 = 0\)
  \item \(x^2y = 0\)
  \item \((x - 1)^2 = 0\)
\end{enumerate}

\item If the roots are \(3\) and \(-4\), the monic polynomial is:

\begin{enumerate}
  \item \(x^2 + x - 12 = 0\)
  \item \(x^2 + x + 12 = 0\)
  \item \(x^2 - 7x + 12 = 0\)
  \item \(x^2 - x - 12 = 0\)
\end{enumerate}

\item The quadratic equation \(6x^2 + 7x + 2 = 0\) has how many real roots?

\begin{enumerate}
  \item \(0\)
  \item \(1\)
  \item \(2\)
  \item \(3\)
\end{enumerate}

\item Which is the equation of a parabola with axis of symmetry \(x = 2\) and passing through \((0,-3)\)?

\begin{enumerate}
  \item \(y = (x-2)^2 - 3\)
  \item \(y = (x-2)^2 - 7\)
  \item \(y = (x-2)^2 + 3\)
  \item \(y = (x-2)^2 + 7\)
\end{enumerate}

\item Which represents the completed square form of \(y = x^2 - 4x + 5\)?

\begin{enumerate}
  \item \(y = (x-2)^2 + 1\)
  \item \(y = (x+2)^2 - 1\)
  \item \(y = (x-2)^2 - 1\)
  \item \(y = (x-4)^2 + 1\)
\end{enumerate}

\item Which of the following is a perfect square trinomial?

\begin{enumerate}
  \item \(x^2 + 6x + 9\)
  \item \(x^2 + 6x + 5\)
  \item \(x^2 - 6x + 9\)
  \item \(x^2 - 6x + 5\)
\end{enumerate}

\end{enumerate}
\end{document}